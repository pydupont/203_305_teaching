\documentclass[12pt]{beamer}

\usepackage{array}
\usepackage{color}
\usepackage{listings}
\usepackage{minted}
\usemintedstyle{autumn}


\renewcommand{\theFancyVerbLine}{\sffamily
\textcolor[rgb]{0.5,0.5,1.0}{\scriptsize
\oldstylenums{\arabic{FancyVerbLine}}}}


\mode<presentation>
{
	%\usetheme{Warsaw}
	%\usetheme{Darmstadt}
	\usetheme{Boadilla}
	%\usetheme{CambridgeUS}
	%\usetheme{Madrid}
}

\makeatletter
\usepackage{times}
\usepackage{tikz}
\usepackage{verbatim}
\usetikzlibrary{arrows,shapes}
\usepackage{natbib} 


\setbeamertemplate{navigation symbols}{
   %\insertslidenavigationsymbol
   %\insertframenavigationsymbol
   %\insertsubsectionnavigationsymbol
   %\insertsectionnavigationsymbol
   %\insertdocnavigationsymbol
   %\insertbackfindforwardnavigationsymbol
}

\title[Python]{Different ways to improve python code efficiency}
\date{}
\institute{}
\author{Pierre-Yves Dupont}

\begin{document}

\begin{frame}
	\titlepage
\end{frame}

\section{Simple example}
\subsection*{}
\begin{frame}
  \begin{block}{}
    \center{Simple example}
  \end{block}
\end{frame}

\begin{frame}[fragile]
  \frametitle{Fibonacci number}
  \begin{block}{fibo.py}
	\begin{minted}[fontsize=\scriptsize, linenos=true, numbersep=5pt]{python}
def fibo(n):
    if n == 0 or n == 1:
        return n
    return fibo(n - 2) + fibo(n - 1)
	\end{minted}
  \end{block} 
        Pure python version: fibo(38) computed in 15s
\end{frame}

\section{Cython}
\subsection*{}
\begin{frame}
  \begin{block}{}
    \center{Cython}
  \end{block}
\end{frame}

\begin{frame}
  \begin{block}{}
    \begin{itemize}[<+->]
    \item apt-get install cython
    \item programming language based on python (Pyrex)
    \item simplify development of C extensions for Python
    \item possible to transform python code in C code
    \item used in Scipy or SAGE
    \end{itemize}
  \end{block}
\end{frame}

\begin{frame}[fragile]
  \begin{block}{Cython on fibo.py}
    \begin{minted}[fontsize=\scriptsize, linenos=true, numbersep=5pt]{bash}
#build fibo.c file
cython fibo.py
#build fibo.so library
gcc fibo.c -o fibo.so -shared -pthread -fPIC \
    -fwrapv -O2 -Wall -fno-strict-aliasing $(pkg-config python --cflags)
    \end{minted}
% $
  \end{block}
  \begin{block}{Using fibo.so}
    \begin{minted}[fontsize=\scriptsize, linenos=true, numbersep=5pt]{python}
import fibo
print fibo.fibo(38)
    \end{minted}
  \end{block}
  Simple Cython version: fibo(38) computed in 10s (33\%)
\end{frame}

\begin{frame}[fragile]
  \frametitle{More improvement}
    \begin{block}{fibo.\emph{pyx}}
       \begin{minted}[fontsize=\scriptsize, linenos=true, numbersep=5pt]{python}
cpdef int fibo(int n):
    if n == 0 or n == 1: 
        return n
    return fibo(n-2) + fibo(n-1)
       \end{minted}
    \end{block}
    \begin{block}{Cython compiling}
    \begin{minted}[fontsize=\scriptsize, linenos=true, numbersep=5pt]{bash}
cython fibo.py
gcc fibo.c -o fibo.so -shared -pthread -fPIC \
    -fwrapv -O2 -Wall -fno-strict-aliasing $(pkg-config python --cflags)
    \end{minted}
% $
  \end{block}
   Cython/pyx version: fibo(38) computed in 5s (66.7\%)
\end{frame}

\begin{frame}[fragile]
  \frametitle{Standalone executable}
  \begin{block}{}
    \begin{minted}[fontsize=\scriptsize, linenos=true, numbersep=5pt]{bash}
cython --embed test_fibo.py -o fibo.c
gcc fibo.c -o fibo $(pkg-config python --cflags) -lpython2.7
./fibo
    \end{minted}
% $
  \end{block}
See more on \url{http://docs.cython.org/index.html}
\end{frame}

\begin{frame}[fragile]
  \frametitle{Pure C version}
  \begin{block}{fibo\_pure\_c.c}
    \begin{minted}[fontsize=\scriptsize, linenos=true, numbersep=5pt]{c}
#include <stdio.h>
#include <stdlib.h>

long fibo(long n){
  if(n == 0 || n == 1){return n;}
  return fibo(n - 2) + fibo(n - 1);
}

void main(int argc, char* argv[]){
  printf("%ld\n",fibo(38));
}
    \end{minted}
  \end{block}
C version: fibo(38) computed in 0.5s
\end{frame}

\section{How to extend python with pure C libraries}
\begin{frame}
  \begin{block}{}
    \center{How to extend python with pure C libraries}
  \end{block}
\end{frame}

\begin{frame}
  \frametitle{SWIG}
  \begin{block}{}
    \begin{itemize}[<+->]
    \item Simplified Wrapper and Interface Generator
    \item Tool used to connect C/C++ libraries to programs written in Python, Perl, Ruby, R, Java\dots
    \item apt-get install swig
    \end{itemize}
  \end{block}
\end{frame}

\begin{frame}[fragile]
  \frametitle{Connect fibo\_pure\_c.c to Python}
  \begin{block}{Interface file fibo\_pure\_c.i}
    \begin{minted}[fontsize=\scriptsize, linenos=true, numbersep=5pt]{c}
%module fibo_pure_c
%{
extern long fibo(long n);
%}
extern long fibo(long n);
    \end{minted}
  \end{block}
  \begin{block}{Wrapper generation}
    \begin{minted}[fontsize=\scriptsize, linenos=true, numbersep=5pt]{bash}
#generate fibo_pure_c_wrap.c
swig -python fibo_pure_c.i
#build the library
gcc -c fibo_pure_c.c fibo_pure_c_wrap.c \
    -I/usr/include/python2.7
ld -shared fibo_pure_c.o fibo_pure_c_wrap.o -o _fibo_pure_c.so 
    \end{minted}
  \end{block}
\end{frame}

\begin{frame}[fragile]
  \begin{block}{Use the C library in python}
    \begin{minted}[fontsize=\scriptsize, linenos=true, numbersep=5pt]{python}
import fibo_pure_c as fibo
print fibo.fibo(38)
    \end{minted}    
  \end{block}
  Python with pure C library: computation of fibo(38) in 0.5s
\end{frame}

\section{Another problem: evaluation of Pi}
\begin{frame}
  \begin{block}{}
    \center{Another problem: evaluation of Pi}
  \end{block}
\end{frame}

\begin{frame}
  \begin{block}{Pi}
    \begin{equation}
      \label{eq:1}
      \pi = \int_0^1 f(x) \, \mathrm dx , with \ f(x) = \frac{4}{1+x^2}
    \end{equation}
    \begin{equation}
      \label{eq:2}
      \pi = \frac{1}{n} \sum_{i=1}^{n} f(x_i) , with\ x_i = \frac{i-\frac{1}{2}}{n}\ for\ i=1,\dots,n
    \end{equation}
  \end{block}
\end{frame}

\begin{frame}[fragile]
    \begin{minted}[fontsize=\scriptsize, linenos=true, numbersep=5pt]{python}
import sys
PI = 3.141592653589793

def f(a):
    return 4.0/(1.0+a**2)

def main():
    while(1):
        n = raw_input("Enter the number of intervals: (0  quits)\n")
        try:
            n=int(n)
        except ValueError:
            return 2
        if n == 0: break
        #LOOP TO PARALLELIZE
        h = 1.0/n
        sum = 0.0
        for i in range(1,n):
            x = h*(i - 0.5)
            sum += f(x)
        pi = h * sum
        #END
        sys.stdout.write("pi is approximatly: %.16f Error is: %.16f \n"\
        % (pi, abs(pi-PI)));
    \end{minted}
$n = 10^8$ pi evaluated in 25s, $n = 10^9$ crash
\end{frame}

\begin{frame}[fragile]
  \frametitle{Weave}
      \begin{minted}[fontsize=\scriptsize, linenos=true, numbersep=5pt]{python}
import sys
from scipy import weave
from scipy.weave import converters
import time
PI = 3.141592653589793

code="""
int i;
double x;
double sum = 0.0;
for(i = 1;i <= n; i++) {
    x = h*((double)i-(double)0.5);
    sum += (double)4.0/((double)1.0+(x*x));
}
return_val = sum;
"""
vars = "h n".split()
...
      \end{minted}
\end{frame}

\begin{frame}[fragile]
  \frametitle{Weave}
      \begin{minted}[fontsize=\scriptsize, linenos=true, numbersep=5pt]{python}
...
def main():
    while(1):
        n = raw_input("Enter the number of intervals: (0  quits)\n")
        try:
            n=int(n)
        except ValueError:
            return 2
        if n == 0: break
        #LOOP TO PARALLELIZE
        h = 1.0/n
        sum = float(0.0)
        tps = time.time()
        sum = weave.inline(code, vars, 
              type_converters = converters.blitz, 
              compiler = 'gcc')
        print time.time() - tps
        pi = h * sum
        #END
        sys.stdout.write("pi is approximatly: %.16f Error is: %.16f \n"\
        % (pi, abs(pi-PI)));
      \end{minted}
$n = 10^8$ pi evaluated in 0.3s, $n = 10^9$ pi evaluated in 4s
\end{frame}

\section{MPI and OpenMP}
\begin{frame}
  \begin{block}{}
    \center{MPI and OpenMP}
  \end{block}
\end{frame}

\begin{frame}
  \frametitle{Parallel computing with MPI and OpenMP}
  \begin{block}{MPI}
    \begin{itemize}[<+->]
    \item Message Passing Interface
    \item Fortran 77, 90, 95 and C/C++
    \item Interfaces for C\#, Java\dots
    \item Parallel machines and on workstation clusters
    \end{itemize}
  \end{block}
\end{frame}

\begin{frame}
  \frametitle{Parallel computing with MPI and OpenMP}
  \begin{block}{OpenMP}
    \begin{itemize}[<+->]
    \item Open MultiProcessing
    \item Fortran 90, 95 and C/C++
    \item Shared memory multiprocessing programming
    \item Most of the processor architectures compatible with MPI
    \item Based on \emph{pragmas} (compiler-specific preprocessor directives)
    \end{itemize}
  \end{block}
\end{frame}

\begin{frame}[fragile]
  \frametitle{Simple C version}
  \begin{minted}[fontsize=\scriptsize, linenos=true, numbersep=5pt]{c}
#include <stdio.h>
#include <stdlib.h>
#include <math.h>
#define PI 3.1415926535897932384626433832795029L

double f(double a) {
  return (double)4.0/((double)1.0+(a*a));
}
...
  \end{minted}
\end{frame}

\begin{frame}[fragile]
  \frametitle{Simple C version}
  \begin{minted}[fontsize=\scriptsize, linenos=true, numbersep=5pt]{c}
int main(int argc, char* argv[])
{
  int n, i;
  double h, pi, sum, x ;
  for(;;) {
    printf("Enter the number of intervals: (0  quits)");
    if(!scanf("%lu",&n)){return 2;}
    if(n ==0)
      break;
    //LOOP TO PARALLELIZE
    h = ((double)1.0)/(double)n;
    sum = 0.0;
    for(i =1;i<=n;i++) {
      x = h*((double)i-(double)0.5);
      sum += f(x);
    }
    pi = h*sum;
    //END
    printf("pi is approximatly: %.16f Error is: %.16f \n",\
    pi, fabs(pi-PI));
  }
  return EXIT_SUCCESS ;
}
$n = 10^9$ pi evaluated in 4sec
  \end{minted}
\end{frame}

\begin{frame}[fragile]
  \frametitle{Pi evaluation using MPI}
  \begin{minted}[fontsize=\scriptsize, linenos=true, numbersep=5pt]{c}
#include <stdio.h>
#include <stdlib.h>
#include <math.h>
#include <mpi.h>
#define PI 3.1415926535897932384626433832795029L

double f(double a) {
  return (double)4.0/((double)1.0+(a*a));
}
...
  \end{minted}
\end{frame}

\begin{frame}[fragile]
  \frametitle{Pi evaluation using MPI}
  \begin{minted}[fontsize=\scriptsize, linenos=true, numbersep=5pt]{c}
int main(int argc, char* argv[])
{
  int n, i;
  double h, pi, sum, x ;
  
  double mypi;
  int myid, numprocs, islave;
  MPI_Status status;
  MPI_Init(&argc, &argv);
  MPI_Comm_size(MPI_COMM_WORLD, &numprocs);
  MPI_Comm_rank(MPI_COMM_WORLD, &myid);
  n=0;
...
  \end{minted}
\end{frame}

\begin{frame}[fragile]
  \begin{minted}[fontsize=\scriptsize, linenos=true, numbersep=5pt]{c}
  for(;;) {
    if (myid == 0){
      printf("Enter the number of intervals: (0  quits)\n"); 
      if(!scanf("%d",&n)){return 2;}
    }
    MPI_Bcast(&n, 1, MPI_INT, 0, MPI_COMM_WORLD);
    if(n ==0)
      break;
    //LOOP TO PARALLELIZE
    h = ((double)1.0)/(double)n;
    sum = 0.0;
    for(i = myid + 1; i <= n; i += numprocs){
      x = h * ((double)i - (double)0.5);
      sum += f(x);
    }
    mypi = h*sum;
    MPI_Reduce(&mypi, &pi, 1, MPI_DOUBLE, MPI_SUM, 0, MPI_COMM_WORLD);
    //END
    if(myid == 0){
      printf("pi is approximatly: %.16f Error is: %.16f \n",\
      pi, fabs(pi-PI));
    }
  }
  MPI_Finalize();
  return EXIT_SUCCESS ;
}
  \end{minted}
\end{frame}

\begin{frame}[fragile]
  \begin{block}{Compilation and Run}
    \begin{minted}[fontsize=\scriptsize, linenos=false, numbersep=5pt]{bash}
mpicc -g -O2 pi_mpi.c -o pi_mpi #mpi_c++ exists
mpirun -np 12 -hostfile hostfile ./pi_mpi
    \end{minted}
  \end{block}
  \begin{block}{}
    \begin{itemize}[<+->]
    \item Different compiler, some differences with standard gnu compiler
    \item The code is completely parallelized
    \item A lot of differences between simple C code and MPI code
    \item $n = 10^9$ pi evaluated in 2s (12 threads)
    \end{itemize}
  \end{block}
\end{frame}

\begin{frame}[fragile]
  \frametitle{Pi evaluation using OpenMP}
  \begin{minted}[fontsize=\scriptsize, linenos=true, numbersep=5pt]{c}
#include <stdio.h>
#include <stdlib.h>
#include <math.h>
#include <omp.h>
#define PI 3.1415926535897932384626433832795029L

double f(double a) {
  return (double)4.0/((double)1.0+(a*a));
}
...
  \end{minted}
\end{frame}

\begin{frame}[fragile]
  \begin{minted}[fontsize=\scriptsize, linenos=true, numbersep=5pt]{c}
int main(int argc, char* argv[]){
  int n, i;
  double h, pi, sum, x ;
  for(;;) {
    printf("Enter the number of intervals: (0  quits)\n");
    if(!scanf("%u",&n)){return 2;}
    if(n ==0)
      break;
    //LOOP TO PARALLELIZE
    h = ((double)1.0)/(double)n;
    sum = 0.0;
#pragma omp parallel for num_threads(12) private(i,x) reduction(+:sum)
    for(i =1;i<=n;i++) {
      x = h*((double)i-(double)0.5);
      //#pragma omp critical
      sum += f(x);
    }
    pi = h*sum;
    //END
    printf("pi is approximatly: %.16f Error is: %.16f \n",\
    pi, fabs(pi-PI));
  }
  return EXIT_SUCCESS ;
}
  \end{minted}
\end{frame}

\begin{frame}[fragile]
  \begin{block}{Compilation and Run}
    \begin{minted}[fontsize=\scriptsize, linenos=false, numbersep=5pt]{bash}
gcc -fopenmp -Wall -O2 pi_openmp.c -o pi_openmp #g++ works fine
./pi_openmp
    \end{minted}
  \end{block}
  \begin{block}{}
    \begin{itemize}[<+->]
    \item Standard compiler, use openmp library
    \item Only part of code (loops) are parallelized
    \item Really close to original C code
    \item $n = 10^9$ pi evaluated in 3s (12 threads)
    \end{itemize}
  \end{block}
\end{frame}

\begin{frame}[fragile]
  \frametitle{Pi evaluation using OpenMP and MPI}
  In MPI version of code, insertion of OMP pragma
  \begin{minted}[fontsize=\scriptsize, linenos=true, numbersep=5pt]{c}
    h = ((double)1.0)/(double)n;
    sum = 0.0;
    #pragma omp parallel for reduction(+:sum) private(i,x)
    for(i = myid + 1; i <= n; i += numprocs){
      x = h * ((double)i - (double)0.5);
      sum += f(x);
    }
    mypi = h*sum;
    MPI_Reduce(&mypi, &pi, 1, MPI_DOUBLE, MPI_SUM, 0, MPI_COMM_WORLD);
  \end{minted}
\end{frame}

\begin{frame}[fragile]
  \begin{block}{Compilation and Run}
    \begin{minted}[fontsize=\scriptsize, linenos=false, numbersep=5pt]{bash}
mpicc -g -O2 -fopenmp pi_mpi.c -o pi_mpi
mpirun -np 12 -hostfile hostfile ./pi_mpi
    \end{minted}
  \end{block}
  \begin{block}{}
    \begin{itemize}[<+->]
    \item MPI compiler
    \item MPI used to share the computation on different nodes
    \item OpenMP used to make threads within a node
    \item $n = 10^9$ pi evaluated in 2s (6 nodes, 2 threads)
    \end{itemize}
  \end{block}
\end{frame}

\begin{frame}[fragile]
  \frametitle{Inlining parallelized C/C++ code in python}
  \begin{minted}[fontsize=\scriptsize, linenos=false, numbersep=5pt]{python}
import sys
from scipy import weave
from scipy.weave import converters
import time
PI = 3.141592653589793

code="""
int i;
double x;
double sum = 0.0;
#pragma omp parallel for private(i,x) reduction(+:sum)
for(i = 1;i <= n; i++) {
    x = h*((double)i-(double)0.5);
    sum += (double)4.0/((double)1.0+(x*x));
}
return_val = sum;
"""
vars = "h n".split()
weave_omp = \
{
    'headers': ['<omp.h>'],
    'extra_compile_args': ['-fopenmp'],
    'extra_link_args': ['-lgomp']
}
...   
  \end{minted}
\end{frame}

\begin{frame}[fragile]
  \frametitle{Inlining parallelized C/C++ code in python}
  \begin{minted}[fontsize=\scriptsize, linenos=false, numbersep=5pt]{python}
def f(a):
    return 4.0/(1.0+a**2)

def main():
    while(1):
        n = raw_input("Enter the number of intervals: (0  quits)\n")
        try:
            n=int(n)
        except ValueError:
            return 2
        if n == 0: break
        #LOOP TO PARALLIZE
        h = 1.0/n
        sum = float(0.0)
        tps = time.time()
        sum = weave.inline(code, vars,
                type_converters = converters.blitz, 
                compiler = 'gcc', **weave_omp)
        print time.time() - tps
        pi = h * sum
        #END
        sys.stdout.write("pi is approximatly: %.16f Error is: %.16f \n" % \
                 (pi, abs(pi-PI)));

  \end{minted}
$n = 10^9$ pi evaluated in 1.3s (12 threads)
\end{frame}

\begin{frame}
  \frametitle{Comparison of the efficiency of the different methods}
  \begin{block}{}
    \begin{tabular}{l l}
slow & 430  seconds \\
numeric & 2.76 seconds \\
Cython pyrex & 2.55 seconds \\
fortran77 & 2.53 seconds \\
fortran90 & 0.60 seconds  \\
fortran95 & 0.59 seconds \\ \hline
C & 2.20 seconds \\
    \end{tabular}
  \end{block}
\end{frame}

\end{document}

%%% Local Variables: 
%%% mode: latex
%%% TeX-master: t
%%% End: 
