\documentclass[article,10pt]{scrartcl}
\usepackage{graphicx}
\usepackage{amsmath}
\usepackage{tikz}
\usepackage{vmargin}
\setmarginsrb{1.3cm}{1.3cm}{1.3cm}{1.3cm}{1.3cm}{1.3cm}{1.3cm}{1.3cm}
\setlength{\parindent}{0cm}
\usepackage{color}
\usepackage{listings}
\usepackage{hyperref}


\hypersetup{
    bookmarks=true,         % show bookmarks bar?
    unicode=false,          % non-Latin characters in Acrobat’s bookmarks
    pdftoolbar=true,        % show Acrobat’s toolbar?
    pdfmenubar=true,        % show Acrobat’s menu?
    pdffitwindow=false,     % window fit to page when opened
    pdfstartview={FitH},    % fits the width of the page to the window
    pdfnewwindow=true,      % links in new window
    colorlinks=true,       % false: boxed links; true: colored links
    linkcolor=black,          % color of internal links (change box color with linkbordercolor)
    citecolor=black,        % color of links to bibliography
    filecolor=black,      % color of file links
    urlcolor=black           % color of external links
}


\usepackage[final]{pdfpages}
% python script syntax taken from 
% http://widerin.org/blog/syntax-highlighting-for-python-scripts-in-latex-documents
\definecolor{Code}{rgb}{0,0,0}
\definecolor{Decorators}{rgb}{0.5,0.5,0.5}
\definecolor{Numbers}{rgb}{0.5,0,0}
\definecolor{MatchingBrackets}{rgb}{0.25,0.5,0.5}
\definecolor{Keywords}{rgb}{0,0,1}
\definecolor{self}{rgb}{0,0,0}
\definecolor{Strings}{rgb}{0,0.63,0}
\definecolor{Comments}{rgb}{0,0.63,1}
\definecolor{Backquotes}{rgb}{0,0,0}
\definecolor{Classname}{rgb}{0,0,0}
\definecolor{FunctionName}{rgb}{0,0,0}
\definecolor{Operators}{rgb}{0,0,0}
\definecolor{Background}{rgb}{0.98,0.98,0.98}

\lstnewenvironment{python}[1][]{
\lstset{
%numbers=left,
%numberstyle=\footnotesize,
%numbersep=1em,
xleftmargin=1em,
framextopmargin=2em,
framexbottommargin=2em,
showspaces=false,
showtabs=false,
showstringspaces=false,
frame=l,
tabsize=3,
% Basic
%basicstyle=\ttfamily\small\setstretch{},
backgroundcolor=\color{Background},
language=Python,
% Comments
commentstyle=\color{Comments}\slshape,
% Strings
stringstyle=\color{Strings},
morecomment=[s][\color{Strings}]{"""}{"""},
morecomment=[s][\color{Strings}]{'''}{'''},
% keywords
morekeywords={import,from,class,def,for,while,if,is,in,elif,else,not,and,or,print,break,continue,return,True,False,None,access,as,,del,except,exec,finally,global,import,lambda,pass,print,raise,try,assert,float},
keywordstyle={\color{Keywords}\bfseries},
% additional keywords
morekeywords={[2]@invariant},
keywordstyle={[2]\color{Decorators}\slshape},
emph={self},
emphstyle={\color{self}\slshape},
%
}}{}

\begin{document}
\title{Read and Plot data with Python - Solutions}
\subtitle{Student and Staff IT Introduction}
\date{27/06/2013}
\author{ssiti2013@gmail.com\footnote{Document written by Pierre-Yves Dupont (p.y.dupont@massey.ac.nz), for IFS staff and students, Massey University.}}
\maketitle

\section{Basic way to read in a CSV file}

\begin{python}
import sys

# sys.argv is a list containing the command line parameters
# sys.argv[0] always contains the name of the python script
if len(sys.argv) != 2:
   print 'Error: give a file name on the command line'
   exit(1) # Exit and return the error code 1

csv_file_name = sys.argv[1]
csv = open(csv_file_name, 'r') #open the csv file
data = []
for line in csv: #read the file
   line = line.rstrip() # remove the 'newline' character at the end of the line
   data.append(float(line))
print data
csv.close() #close the csv file
\end{python}

\section{Read a complex CSV file using \textbf{numpy} library.}
\begin{python}
import sys
import numpy

if len(sys.argv) != 2:
   print 'Error: give a file name on the command line'
   exit(1) # Exit and return the error code 1

data = numpy.genfromtxt(sys.argv[1], dtype=float, delimiter=',', skip_header=True) 
print 'All data'
print data
print 'Column 1'
print data[:,0] # column 1
print 'Column 1'
print data[:,1] # column 2
\end{python}
\newpage
\section{Simple plot of a data from a CSV file containing only one column.}
\begin{python}
import matplotlib.pyplot as plt # import the pyplot module as plt
import sys

# sys.argv is a list containing the command line parameters
# sys.argv[0] always contains the name of the python script
if len(sys.argv) != 2:
   print 'Error: give a file name on the command line'
   exit(1) # Exit and return the error code 1

csv_file_name = sys.argv[1]
csv = open(csv_file_name, 'r') #open the csv file
data = [] #initiation of the data list
for line in csv: #read the file
   line = line.rstrip() # remove the 'newline' character at the end of the line
   data.append(float(line))#append the current value in the data list
print data
csv.close() #close the csv file

plt.plot(data) #Simple scatter plot of the data
plt.ylabel('My data') #title of the Y axis
plt.show() # show the plot
\end{python}

\section{Scatter plot of data from a CSV file}
\begin{python}
import matplotlib.pyplot as plt
import sys
import numpy

if len(sys.argv) != 2:
   print 'Error: give a file name on the command line'
   exit(1) # Exit and return the error code 1

data = numpy.genfromtxt(sys.argv[1], dtype=float, delimiter=',', skip_header=True) 

x = data[:,0]
y = data[:,1]

plt.plot(x,y, 'bo')
plt.xlabel( "X values" )
plt.ylabel( "Y values" )
plt.show()
\end{python}
\newpage{}
\section{Box plot of data from a CSV file}
\begin{python}
import matplotlib.pyplot as plt
import sys
import numpy

if len(sys.argv) != 2:
   print 'Error: give a file name on the command line'
   exit(1) # Exit and return the error code 1

data = numpy.genfromtxt(sys.argv[1], dtype=float, delimiter='\t', skip_header=True) 

plt.boxplot(data)
plt.ylim([-1,4])
plt.show()
\end{python}
\end{document}
